%% Pour compiler, on utilise XeLateX avec Ctrl-Shift-F1
%% �a produit une image PNG

\documentclass[convert={ghostscript, density = 1000}]{standalone}

\usepackage{pstricks-add}

\begin{document}

\psset{algebraic,unit=1}
\begin{pspicture}(-2,-0.5)(4.5,4)

\psaxes[labels=none,ticks=none,arrowscale= 1.5]{->}(0,0)(-2,-0.5)(4,3.5)

\Polar

\pscurve[fillstyle=solid,fillcolor=lightgray,linecolor=lightgray](3,22)(2.9,40)(3.2,50)(3.3,65)(2.9,80)(2.7,100)(3,110)(3.1,130)
\psline[fillstyle=solid,fillcolor=lightgray,linecolor=lightgray](0,0)(3,22)(3.1,130)(0,0)
\pscurve[linewidth=2pt](3,22)(2.9,40)(3.2,50)(3.3,65)(2.9,80)(2.7,100)(3,110)(3.1,130)
\psline[linewidth=2pt](0,0)(3,22)
\psline[linewidth=2pt](3.1,130)(0,0)
% \psaxes[labels=none,ticks=none,arrowscale= 1.5]{->}(0,0)(-2,-0.5 )(4,3.5)
\psline[linewidth=.5pt](0,0)(2.9,40)
\psline[linewidth=.5pt](0,0)(3.2,50)
\psline[linewidth=.5pt](0,0)(3.3,65)
\psline[linewidth=.5pt](0,0)(2.9,80)
\psline[linewidth=.5pt](0,0)(2.7,100)
\psline[linewidth=.5pt](0,0)(3,110)

\psline[linestyle=dotted](3.3,65)(3.6,65)
\psline[linestyle=dotted](2.9,80)(3.2,80)
\psline[linestyle=dashed](0,72)(3.05,72)
\psarc{-}{3.05}{65}{80}
\uput{0.1}[65]{0}(3.6,65){$\theta_{i-1}$}
\uput{0.1}[80]{0}(3.2,80){$\theta_{i}$}
\psline[arrowscale= 1.5]{<-}(2.05,72)(4,30)
\uput{0.1}[0]{0}(4,30){$f(\theta_{i}^*)$}
	
\end{pspicture}

\end{document}