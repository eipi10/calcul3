\documentclass[]{book}
\usepackage{lmodern}
\usepackage{amssymb,amsmath}
\usepackage{ifxetex,ifluatex}
\usepackage{fixltx2e} % provides \textsubscript
\ifnum 0\ifxetex 1\fi\ifluatex 1\fi=0 % if pdftex
  \usepackage[T1]{fontenc}
  \usepackage[utf8]{inputenc}
\else % if luatex or xelatex
  \ifxetex
    \usepackage{mathspec}
  \else
    \usepackage{fontspec}
  \fi
  \defaultfontfeatures{Ligatures=TeX,Scale=MatchLowercase}
\fi
% use upquote if available, for straight quotes in verbatim environments
\IfFileExists{upquote.sty}{\usepackage{upquote}}{}
% use microtype if available
\IfFileExists{microtype.sty}{%
\usepackage{microtype}
\UseMicrotypeSet[protrusion]{basicmath} % disable protrusion for tt fonts
}{}
\usepackage[margin=1in]{geometry}
\usepackage{hyperref}
\hypersetup{unicode=true,
            pdftitle={Calcul différentiel et intégral dans l'espace},
            pdfauthor={Marc-André Désautels},
            pdfborder={0 0 0},
            breaklinks=true}
\urlstyle{same}  % don't use monospace font for urls
\usepackage{natbib}
\bibliographystyle{apalike}
\usepackage{longtable,booktabs}
\usepackage{graphicx,grffile}
\makeatletter
\def\maxwidth{\ifdim\Gin@nat@width>\linewidth\linewidth\else\Gin@nat@width\fi}
\def\maxheight{\ifdim\Gin@nat@height>\textheight\textheight\else\Gin@nat@height\fi}
\makeatother
% Scale images if necessary, so that they will not overflow the page
% margins by default, and it is still possible to overwrite the defaults
% using explicit options in \includegraphics[width, height, ...]{}
\setkeys{Gin}{width=\maxwidth,height=\maxheight,keepaspectratio}
\IfFileExists{parskip.sty}{%
\usepackage{parskip}
}{% else
\setlength{\parindent}{0pt}
\setlength{\parskip}{6pt plus 2pt minus 1pt}
}
\setlength{\emergencystretch}{3em}  % prevent overfull lines
\providecommand{\tightlist}{%
  \setlength{\itemsep}{0pt}\setlength{\parskip}{0pt}}
\setcounter{secnumdepth}{5}
% Redefines (sub)paragraphs to behave more like sections
\ifx\paragraph\undefined\else
\let\oldparagraph\paragraph
\renewcommand{\paragraph}[1]{\oldparagraph{#1}\mbox{}}
\fi
\ifx\subparagraph\undefined\else
\let\oldsubparagraph\subparagraph
\renewcommand{\subparagraph}[1]{\oldsubparagraph{#1}\mbox{}}
\fi

%%% Use protect on footnotes to avoid problems with footnotes in titles
\let\rmarkdownfootnote\footnote%
\def\footnote{\protect\rmarkdownfootnote}

%%% Change title format to be more compact
\usepackage{titling}

% Create subtitle command for use in maketitle
\newcommand{\subtitle}[1]{
  \posttitle{
    \begin{center}\large#1\end{center}
    }
}

\setlength{\droptitle}{-2em}

  \title{Calcul différentiel et intégral dans l'espace}
    \pretitle{\vspace{\droptitle}\centering\huge}
  \posttitle{\par}
    \author{Marc-André Désautels}
    \preauthor{\centering\large\emph}
  \postauthor{\par}
      \predate{\centering\large\emph}
  \postdate{\par}
    \date{2018-09-19}

\usepackage{booktabs}

\hypersetup{colorlinks=true, urlcolor=blue}

\renewcommand{\chaptername}{Chapitre}
\renewcommand{\contentsname}{Table des Matières}
\renewcommand{\partname}{Partie}
\renewcommand\bibname{Bibliographie}

\usepackage{amsthm}
\newtheorem{theorem}{Théorème}[chapter]
\newtheorem{lemma}{Lemme}[chapter]
\theoremstyle{definition}
\newtheorem{definition}{Définition}[chapter]
\newtheorem{corollary}{Corollaire}[chapter]
\newtheorem{proposition}{Proposition}[chapter]
\theoremstyle{definition}
\newtheorem{example}{Exemple}[chapter]
\theoremstyle{definition}
\newtheorem{exercise}{Exercice}[chapter]
\theoremstyle{remark}
\newtheorem*{remark}{Remarque}
\newtheorem*{solution}{Solution}
\let\BeginKnitrBlock\begin \let\EndKnitrBlock\end
\begin{document}
\maketitle

{
\setcounter{tocdepth}{1}
\tableofcontents
}
\hypertarget{introduction}{%
\chapter*{Introduction}\label{introduction}}
\addcontentsline{toc}{chapter}{Introduction}

\hypertarget{a-propos-de-ce-document}{%
\section*{À propos de ce document}\label{a-propos-de-ce-document}}
\addcontentsline{toc}{section}{À propos de ce document}

\hypertarget{remerciements}{%
\subsection*{Remerciements}\label{remerciements}}
\addcontentsline{toc}{subsection}{Remerciements}

Ce document est généré par l'excellente extension
\href{https://bookdown.org/}{bookdown} de
\href{https://yihui.name/}{Yihui Xie}.

\hypertarget{license}{%
\subsection*{License}\label{license}}
\addcontentsline{toc}{subsection}{License}

Ce document est mis à disposition selon les termes de la
\href{http://creativecommons.org/licenses/by-nc-sa/4.0/}{Licence
Creative Commons Attribution - Pas d'Utilisation Commerciale - Partage
dans les Mêmes Conditions 4.0 International}.

\begin{figure}
\centering
\includegraphics{resources/icons/license_cc.png}
\caption{Licence Creative Commons}
\end{figure}

\hypertarget{taylor}{%
\chapter{Les séries de Taylor}\label{taylor}}

\hypertarget{les-polynomes-de-taylor-et-de-maclaurin}{%
\section{Les polynômes de Taylor et de
MacLaurin}\label{les-polynomes-de-taylor-et-de-maclaurin}}

De tous les types de fonctions, les fonctions polynomiales sont celles
qui se dérivent et s'intègrent le plus facilement. De plus, si leur
degré est inférieur ou égal à 5, des formules permettent de trouver
facilement leurs zéros. Pour ces raisons, l'écriture d'une fonction
\(f(x)\) sous la forme d'un polynôme de degré \(n\), \(P_n(x)\), nous
permet de l'étudier aisément. Cependant, en écrivant une fonction sous
la forme d'un polynôme, nous obtenons une approximation.

L'approche de Taylor et de MacLaurin est couramment utilisée pour
transformer une fonction en polynôme.

\hypertarget{les-polynomes-de-maclaurin}{%
\subsection{Les polynômes de
MacLaurin}\label{les-polynomes-de-maclaurin}}

Pour savoir de quelle manière exprimer une fonction \(f(x)\) sous la
forme d'un polynôme, nous étudierons un cas particulier des polynômes de
Taylor, soit les polynômes de MacLaurin.

\BeginKnitrBlock{definition}[Polynôme de Maclaurin]
\protect\hypertarget{def:unnamed-chunk-1}{}{\label{def:unnamed-chunk-1}
\iffalse (Polynôme de Maclaurin) \fi{} }Soit \(f(x)\) une fonction
dérivable au moins \(n\) fois. Le \textbf{polynôme de MacLaurin} de
degré \(n\), \(P_n(x)\), de la fonction \(f(x)\) est un polynôme
satisfaisant les conditions suivantes: \begin{align}
\begin{split}
f(0) &= P_n(0) \\
\left.\dfrac{d^k f}{dx^k}\right|_{x=0} &= \left.\dfrac{d^k P_n}{dx^k}\right|_{x=0}, \quad \text{pour } k\in\{1,\ldots,n\}
\end{split}
\end{align}
\EndKnitrBlock{definition}

Les deux conditions suivantes permettent de construire le polynôme de
MacLaurin pour une fonction \(f(x)\) quelconque. Nous savons qu'un
polynôme de degré \(n\) s'écrit de la façon suivante:

\begin{align*}
P_n(x)=a_0+a_1x+a_2x^2+...+a_{n-1}x^{n-1}+a_nx^n
\end{align*}

Pour trouver les coefficients \(a_k\), nous devons obtenir les dérivées
successives de \(P_n(x)\). Ainsi:

\begin{align}
\begin{split}
P_n(x) &= a_0+a_1x+a_2x^2+...+a_{n-1}x^{n-1}+a_nx^n \\
P_n^{(1)}(x) &= 1a_1+2a_2x+3a_3x^2+...+(n-1)a_{n-1}x^{n-2}+na_nx^{n-1} \\
P_n^{(2)}(x) &= 2\cdot 1a_2+3\cdot 2a_3x+...+(n-1)(n-2)a_{n-1}x^{n-3}+n(n-1)a_nx^{n-2} \\
P_n^{(3)}(x) &= 3\cdot 2\cdot 1a_3+...+(n-1)(n-2)(n-3)a_{n-1}x^{n-4}+n(n-1)(n-2)a_nx^{n-3}
\end{split}
\end{align}

et ainsi de suite.

Par définition, nous savons que \(f(0)=P_n(0)\). Ainsi:

\begin{align}
\begin{split}
f(0)&=P_n(0) \\
f(0) &= a_0+a_1(0)+a_2(0)^2+...+a_{n-1}(0)^{n-1}+a_n(0)^n \\
f(0)&=a_0
\end{split}
\end{align}

De même, nous savons que \(f^{(1)}(0)=P_n^{(1)}(0)\). Ainsi:

\begin{align}
\begin{split}
f^{(1)}(0)&=P_n^{(1)}(0)\\
f^{(1)}(0)&=1a_1+2a_2(0)+3a_3(0)^2+...+(n-1)a_{n-1}(0)^{n-2}+na_n(0)^{n-1} \\
f^{(1)}(0)&=1a_1  \\
\dfrac{f^{(1)}(0)}{1}&=a_1
\end{split}
\end{align}

De la même façon, nous savons que \(f^{(2)}(0)=P_n^{(2)}(0)\). Ainsi:

\begin{align}
\begin{split}
f^{(2)}(0)&=P^{(2)}_n(0)\\
f^{(2)}(0)&=2\cdot 1a_2+3\cdot 2a_3(0)+\ldots+(n-1)(n-2)a_{n-1}(0)^{n-3}+n(n-1)a_n(0)^{n-2} \\
f^{(2)}(0)&=2\cdot 1a_2  \\
\dfrac{f^{(2)}(0)}{2\cdot 1}&=a_2
\end{split}
\end{align}

D'une manière générale, nous trouvons:

\begin{align}
\begin{split}
a_k &= \dfrac{1}{k\cdot (k-1)\cdot ...\cdot 3\cdot 2\cdot 1}f^{(k)}(0) \\
&= \dfrac{f^{(k)}(0)}{k!}
\end{split}
\end{align}

\BeginKnitrBlock{remark}[Factorielle]
\iffalse{} {Remarque (Factorielle). } \fi{}La factorielle d'un nombre
entier \(k\) positif, notée \(k!\), est égale à: \begin{equation*}
k! = k(k-1)(k-2)\cdot\ldots\cdot 3\cdot 2\cdot 1
\end{equation*} Et par définition \(0!=1\).
\EndKnitrBlock{remark}

Nous obtenons donc une équation pour déterminer le polynôme de MacLaurin
d'une fonction.

\BeginKnitrBlock{definition}[Polynôme de MacLaurin]
\protect\hypertarget{def:unnamed-chunk-3}{}{\label{def:unnamed-chunk-3}
\iffalse (Polynôme de MacLaurin) \fi{} }Soit \(f(x)\) une fonction
dérivable au moins \(n\) fois en \(x=0\). Le \textbf{polynôme de
MacLaurin} de degré \(n\), \(P_n(x)\), est donné par: \begin{align*}
P_n(x)&=\sum_{k=0}^n\dfrac{f^{(k)}(0)}{k!}x^k=f(0)+f^{(1)}(0)x+\dfrac{f^{(2)}(0)}{2!}x^2+...+\dfrac{f^{(n)}(0)}{n!}x^n
\end{align*}
\EndKnitrBlock{definition}

\BeginKnitrBlock{example}
\protect\hypertarget{exm:unnamed-chunk-4}{}{\label{exm:unnamed-chunk-4}
}Trouvez les polynômes de MacLaurin de degrés 1, 2 et 3 de \(f(x)=e^x\).
\EndKnitrBlock{example}

\hypertarget{edo}{%
\chapter{Les équations différentielles ordinaires}\label{edo}}

\hypertarget{coordpolaires}{%
\chapter{Les coordonnées polaires}\label{coordpolaires}}

\hypertarget{fctvar}{%
\chapter{Les fonctions de plusieurs variables}\label{fctvar}}

\hypertarget{intfct}{%
\chapter{L'intégration de fonctions de plusieurs
variables}\label{intfct}}

\bibliography{book.bib,packages.bib}


\end{document}
